%!TEX root = ../Thesis.tex

%% TIKZ
\usetikzlibrary{decorations.pathreplacing}
\usetikzlibrary{decorations.pathmorphing}
\usetikzlibrary{calc}
\usetikzlibrary{shapes.geometric}
\usetikzlibrary{shapes.misc}
\usetikzlibrary{positioning}
\usetikzlibrary{patterns}
\usetikzlibrary{fit}
\usetikzlibrary{arrows}
\usetikzlibrary{decorations.markings}
\usetikzlibrary{matrix}
\tikzstyle{every picture}+=[remember picture]


\usepackage[english]{babel}
\usepackage{upgreek}
\usepackage{amsthm,amsmath,nicefrac,graphicx,lastpage,array}
%,amssymb
 

\newtheorem{theorem}{Theorem}[section]
\newtheorem{corollary}{Corollary}[theorem]
\newtheorem{lemma}[theorem]{Lemma}
\newtheorem{definition}{Definition}
\newtheorem{fact}{Fact}


\frenchspacing


 
%%%%%%%%%%%%%%%%%%%%%%%%%%%%%%%%%%%%%%%%%%%
% Proper and meaningful colored Fixmes.
% Simpler interface to fixmes/todos
\usepackage{soul} % provides sethlcolor, hl for fixes
\usepackage{framed} % shading for notes
\definecolor{shadecolor}{rgb}{1,0.8,0.2} % shade color for notes
%\definecolor{shadecolor}{rgb}{1,0.8,0.2} % shade color for notes

\usepackage[inline,nomargin,author=,draft]{fixme}
\fxusetheme{color}
%\fxsetface{inline}{\bfseries}
%\fxsetface{inline}{\rmfamily}
%\newcommand{\note}[1]{\fxnote{\hlc[gray!0.2]{#1}}}
\newcommand{\note}[1]{\begin{shaded}\fxnote{#1}\end{shaded}}
\renewcommand{\todo}[1]{\begin{shaded}\fxfatal{#1}\end{shaded}}

\newcommand{\hlc}[2][yellow]{{\sethlcolor{#1} \hl{#2}}}
\newcommand{\fix}[1]{\fxfatal{\hlc[yellow]{#1}}}
\newcommand{\docite}[1]{\footnote{\fix{#1}}}
%%%%%%%%%%%%%%%%%%%%%%%%%%%%%%%%%%%%%%%%%%%




%%%%%%%%%%%%%%%%%%%%%%%%%%%%%%%%%%%%%%%%%%%
%%%%%%%%%%%%%%%%%% Info about papers
\newenvironment{infosection}
    {}
    {\clearpage}
\newcommand{\published}[1]{\noindent {\paragraph{Publication} #1}}
%\newenvironment{authors}
%    {\begin{center}}
%    {\end{center}\setcounter{footnote}{0}\vspace{1em}}
    \newenvironment{authors}
      {\vspace{1cm}\par\edef\savedfootnotenumber{\number\value{footnote}}
       \renewcommand{\thefootnote}{\fnsymbol{footnote}}
       \setcounter{footnote}{0}\centering}
      {\par\setcounter{footnote}{\savedfootnotenumber}}
    
\newenvironment{uninames}
    {\begin{center}}
    {\end{center}\vspace{3em}}



\makeatletter
\def\@xfootnote[#1]{%
  \protected@xdef\@thefnmark{#1}%
  \@footnotemark\@footnotetext}
\makeatother

%% Specify affiliations with this massive hack
%\newcommand{\uniname}[2]{\footnotemark[#1] #2 \\}
\newcommand{\uni}[1]{$^{#1}$}
\newcommand*\samethanks[1][\value{footnote}]{\footnotemark[#1]}
%%%%%%%%%%%%%%%%%%%%%%%%%%%%%%%%%%%%%%%%%%%


% Table row height. From IEEE package documentation
\renewcommand{\arraystretch}{1.2}


%%%%%%%%%%%%%%% MAKES AUTHORS ON TOP WITH MAKETITLE.
\begin{comment}
%%% FROM LIPICS
\makeatletter
\renewcommand\maketitle{\par
  \begingroup
    \renewcommand\thefootnote{\@fnsymbol\c@footnote}%
    \if@twocolumn
      \ifnum \col@number=\@ne
        \@maketitle
      \else
        \twocolumn[\@maketitle]%
      \fi
    \else
      %\newpage
      \global\@topnum\z@   % Prevents figures from going at top of page.
      \@maketitle
    \fi
    \thispagestyle{plain}\@thanks
  \endgroup
  \setcounter{footnote}{0}%
  \global\let\thanks\relax
  \global\let\maketitle\relax
  \global\let\@maketitle\relax
  \global\let\@thanks\@empty
  \global\let\@author\@empty
  \global\let\@date\@empty
  \global\let\@title\@empty
  \global\let\title\relax
  \global\let\author\relax
  \global\let\date\relax
  \global\let\and\relax
}
\def\@maketitle{%
  %\newpage
  %\null\vskip-\baselineskip
  %\vskip-\headsep
  \let \footnote \thanks
  \parindent\z@ \raggedright
    \ifnum\c@authors=0 %
      \@latexerr{No \noexpand\author given}%
        {Provide at least one author. See the LIPIcs class documentation.}%
    \else
      \@author
    \fi
  \par}
\makeatother

\usepackage{authblk}
\renewcommand*\Authand{{ and }}
\end{comment}

