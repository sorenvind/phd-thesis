%!TEX root = ../Thesis.tex
\chapter{Preface}
This dissertation presents results in the general topic of algorithms and data structures. 
It was prepared during my enrollment from October 2012 to March 2015 as a PhD student at the Department of Applied Mathematics and Computer Science in partial fulfillment of the requirements for obtaining a PhD degree at the Technical University of Denmark.
%It is the result of my work within (mostly) theoretical computer science, and more specifically in the area of algorithms and data structures research.
%It contains a basic introduction to the field of algorithms and data structures and a description of the results I have obtained during my PhD, followed by seven original included papers. 
%, which I consider the fundamental area for efficiently solving problems in computer science. 

% from October 2012 to March 2015
My primary focus has been on designing solutions for problems using little space. Each included paper contain at least one data structure with this particular property that solve a problem on strings, points or integers.
% though they span several subfields, .%, though they span several subfields of algorithmic research, namely combinatorial pattern matching, geometry, and compression. 
%The papers are the full body of research I have produced during my enrollment from October 2012 to March 2015. 
Five papers have been peer-reviewed and published at international conferences and the remaining two are in submission.


\paragraph{Managed Video as a Service.}
My PhD scholarship was part of project \emph{Managed Video as a Service} which was conceived in collaboration by Aalborg University, the Technical University of Denmark and industry partners Milestone Systems, Nabto and Securitas. The project was funded by a grant from the Danish National Advanced Technology Foundation (H{\o}jteknologifonden), scheduled to last three years and involve four PhD students.

My part of the project was called \emph{Algorithms for Metadata}. The goal was to research and develop data structures for indexing massive amounts of meta data that support efficient queries. One papers is a direct result of my involvement in the project and collaboration with Milestone Systems. %This paper also presents the only prototype and experimental results included in the thesis. The remaining papers have a more theoretical focus.


\paragraph{Acknowledgements.}
I am particularly grateful to my dear advisors Philip Bille and Inge Li Gørtz for suggesting that this particular PhD position would be a great fit for me, thereby pulling me back into the academic world that I had decided to leave. Your advice have been indispensable and I hope you are not too dissappointed in my decision to leave academia. 
My great family and friends, colleagues, office mates and the rest of the Copenhagen Algorithms community have made the last two and a half years very enjoyable. 
I also owe thanks to Roberto Grossi, Raphaël Clifford and Benjamin Sach for accepting me as a guest during my research visits in Pisa and Bristol: you made me feel at home.
My coauthors Philip Bille, Patrick Hagge Cording, Roberto Grossi, Inge Li Gørtz, Markus Jalsenius, Giulia Menconi, Nadia Pisanti, Benjamin Sach, Frederik Rye Skjoldjensen, Roberto Trani, and Hjalte Wedel Vildhøj made the periods of intense research insanely rewarding. And finally, thanks to Lærke for her support, love and understanding.\\
~\\
\noindent Thank you all. I take all responsibility, but you made this possible.


\vfill
\begin{flushright}
    \thesisauthor{}\\
    \thesislocation{}, March \the\year
\end{flushright}

%\todo{Write Acks}
%\begin{itemize}
%    \item advisors
%    \item external visit
%    \item friends and family
%    \item office
%    \item section
%    \item institute
%    \item university
%    \item coauthors: Philip Bille, Patrick Hagge Cording, Roberto Grossi, Inge Li Gørtz, Markus Jalsenius, Giulia Menconi, Nadia Pisanti, Benjamin Sach, Frederik Rye Skjoldjensen, Roberto Trani, Hjalte Wedel Vildhøj
%\end{itemize}

 

